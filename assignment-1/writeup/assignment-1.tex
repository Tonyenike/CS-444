\documentclass[10pt,english]{article}
\usepackage{times}
\usepackage{cite}
\usepackage[T1]{fontenc}
\usepackage[english]{babel}
\usepackage{longtable, hyperref}
\usepackage{listings}
\usepackage[margin = .75 in]{geometry}
\usepackage[utf8]{inputenc}
\newcommand{\longtableendfoot}{Please continue at the next page}
%this is a comment
\title{Assignment 1 - Getting acquainted \& Concurrency}
\author{Nickoli Londura \& Benjamin Martin - Group 7 \\ Fall 2018 \\ CS 444 \\ Oregon State University}


% [language=C]

\begin{document}
\maketitle
% \tableofcontents

\begin{abstract}

\noindent This is the assignment 1 write-up for group 7 of CS 444. This assignment combines the "Getting acquainted" section and "concurrency" sections on Canvas. We explore how to build a kernel and run a VM on it on the OS2 server, and we review how to run concurrent threads and apply mutual exclusion (mutexes) on the threads that need to use shared resources.

\end{abstract}

\newpage


\section{Step-by-step walkthrough}
\par This is how we were able to run the VM on our own built kernel.

\begin{enumerate}
\item Type \begin{verbatim} mkdir group7 \end{verbatim}  in the \begin{verbatim} /scratch/fall2018 \end{verbatim} folder on the os2 server. For the TA grading this, you may type \begin{verbatim} mkdir spoofgroup \end{verbatim} instead so that you don't overwrite our folder!
\item Move into the group 7 directory by typing \begin{verbatim} cd group7 \end{verbatim}
\item In this directory, obtain the linux-yocto by typing \begin{verbatim} git clone git://git.yoctoproject.org/linux-yocto-3.19 \end{verbatim}
\item Move into the new linux-yocto directory. \begin{verbatim} cd linux-yocto-3.19 \end{verbatim}
\item Change the version by typing \begin{verbatim} git checkout tags/v3.19.2 \end{verbatim}
\item Copy the following files into the current folder using the cp command:
\begin{verbatim} cp -f /scratch/files/environment-setup-i586-poky-linux ./ \end{verbatim} 
\begin{verbatim} cp -f /scratch/files/environment-setup-i586-poky-linux.csh ./ \end{verbatim}
\begin{verbatim} cp -f /scratch/files/config-3.19.2-yocto-standard ./ \end{verbatim}
\begin{verbatim} cp -f /scratch/files/core-image-lsb-sdk-qemux86.ext4 ./ \end{verbatim}
\item Rename the config file
\begin{verbatim} mv -f ./config-3.19.2-yocto-standard ./.config \end{verbatim}
\item Build the kernel.
\begin{verbatim} make -j4 all \end{verbatim}
\item Source the environment files (do both just in case)
\begin{verbatim} source environment-setup-i586-poky-linux \end{verbatim}
\begin{verbatim} source environment-setup-i586-poky-linux.csh \end{verbatim}
\item Start the VM on our newly built kernel. We are group 7, so our port number was 5507.
\begin{verbatim}
qemu-system-i386 -gdb tcp::5507 -S -nographic -kernel ./arch/i386/boot/bzImage 

-drive file=core-image-lsb-sdk-qemux86.ext4,if=virtio -enable-kvm 

-net none -usb -localtime --no-reboot 

--append "root=/dev/vda rw console=ttyS0 debug"    \end{verbatim} 
\item Start a new session on os2 WHILE KEEPING THIS CURRENT SESSION OPEN. I will assume that you know how to do this.
\item On the new session, type \begin{verbatim} gdb \end{verbatim}
\item Connect to the kernel, type \begin{verbatim} target remote localhost:5507 \end{verbatim}
\item Finish kernel boot-up, type \begin{verbatim} continue \end{verbatim}
\item Switch back to original session. Login with credentials (root and no password)
\item We are running the VM on our built kernel. Finished. Type \begin{verbatim} reboot \end{verbatim} to get out of there.
\item To test only the VM on the prebuilt kernel, type

\begin{verbatim}
qemu-system-i386 -gdb tcp::5507 -S -nographic -kernel bzImage-qemux86.bin 

-drive file=core-image-lsb-sdk-qemux86.ext4,if=virtio -enable-kvm 

-net none -usb -localtime --no-reboot 

--append "root=/dev/vda rw console=ttyS0 debug"    \end{verbatim} 
\end{enumerate}

\section{qemu flags definitions}


-gdb: Debug Mode                          \\
-S: Manual CPU Start                      \\
-nographic: Disable Graphics              \\
-kernel: Kernel File                      \\
-drive: Generic Drive                     \\
-enable-kvm: KVM Virtualization           \\
-net: Network Options                     \\
-usb: Enable USB                          \\
-localtime: Sets the real time clock (RTC) to local. Been replaced with -rtc [options] \\
--no-reboot: No Reboot                                                                 \\
--append: Command Line Parameter                                                       \\

\section{Concurrency Questions}

\textbf{1. What do you think the main point of this assignment is?}
Review concurrency in C and learn a little bit about OS random number generation. This helps us get familiar with operations of the OS and how they are coded in C. This exploratory project helped us have a deeper understanding of the essential properties of the C language for concurrency problems. C really makes things easy to implement with the mutex. \\


\noindent \textbf{2. How did you personally approach the problem? Design decisions, algorithm, etc.}
Solved the problem by function. Sort the task into producer versus consumer. The buffer array was implemented as a dynamic array which was a queue (FIFO). Mutex locks were placed around anytime the buffer data was accessed or modified (which covered a majority of the producer and consumer function).  \\


\noindent \textbf{3. How did you ensure your solution was correct? Testing details, for instance.}
Debugging was lots and lots of trace statements. It's how we debug all our projects in CS because we are pros. We also checked wait time outputs to ensure they were in range and checked mutex locks by commenting out unlock lines to see that the code would halt. This ensures that the buffer is being locked by the threads.\\


\noindent \textbf{4. What did you learn?}
Re-learned how to use mutex and create threads in C. This stuff is mostly review from OS1, since we had covered thread creation and mutual exclusion in OS1. Ben Brewster did a good job. We learned about random number generation, such as the Mersene Twister and the rdrand call.\\


\section{Version Control Log}

%% This file was generated by the script latex-git-log
\begin{tabular}{lp{12cm}}
  \label{tabular:legend:git-log}
  \textbf{acronym} & \textbf{meaning} \\
  V & \texttt{version} \\
  tag & \texttt{git tag} \\
  MF & Number of \texttt{modified files}. \\
  AL & Number of \texttt{added lines}. \\
  DL & Number of \texttt{deleted lines}. \\
\end{tabular}

\bigskip

% \iflanguage{ngerman}{\shorthandoff{"}}{}
\begin{longtable}{|rlllrrr|}
\hline \multicolumn{1}{|c}{\textbf{V}} & \multicolumn{1}{c}{\textbf{tag}}
& \multicolumn{1}{c}{\textbf{date}}
& \multicolumn{1}{c}{\textbf{commit message}} & \multicolumn{1}{c}{\textbf{MF}}
& \multicolumn{1}{c}{\textbf{AL}} & \multicolumn{1}{c|}{\textbf{DL}} \\ \hline
\endhead

\hline \multicolumn{7}{|r|}{\longtableendfoot} \\ \hline
\endfoot

\hline% \hline
\endlastfoot

\hline 1 &  & 57351b4b4272e2e202f9c37236e6aa529d57d48d 2018-10-04 & Initial test & 4 & 118 & 0 \\
\hline 2 &  & 457cb43d9a5ffd145692bf8048d9f67837ef5b67 2018-10-04 & Create README.md & 1 & 1 & 0 \\
\hline 3 &  & a0d31694c3643232027d1603ce90157f0aa4f7d0 2018-10-04 & Create README.md & 1 & 1 & 0 \\
\hline 4 &  & 7ed32fd0a03d4cd228a052dc6e82d18036585619 2018-10-04 & Small change & 0 & 0 & 0 \\
\hline 5 &  & 789c79dd6550916482399238babfdb32c9a9bc26 2018-10-04 & Added assignment folders & 6 & 111 & 111 \\
\hline 6 &  & 73e30d4da7ba8a16076476ad9dbc6762dabe5842 2018-10-04 & Added assignment folders & 6 & 111 & 111 \\
\hline 7 &  & f61afcd4df3637188959edd4269d9964eb01196a 2018-10-04 & Organized & 10 & 4402 & 111 \\
\hline 8 &  &   & 0 & 0 & 0 \\
\hline 9 &  &   & 0 & 0 & 0 \\
\hline 10 &  &   & 0 & 0 & 0 \\
\hline 11 &  & fa83ec19fc9d6e1c06b20b5edc4ee75f4a0d7b3a 2018-10-04 & There we go & 0 & 0 & 0 \\
\hline 12 &  & 06539913bd0fab683fe6cde3bd06cde97e9431f3 2018-10-10 & Added concurrency code & 1 & 35 & 0 \\
\hline 13 &  & b5310109d285d732ccc2527efe50f124dc0415bc 2018-10-15 & Added stuff & 6 & 57 & 4306 \\
\hline 14 &  & 840df6290e155004b32584878ee1d12810aeff5e 2018-10-15 & Finished concurrency assignment & 11 & 2102 & 57 \\
\hline 15 &  & 35987c988887860b9d57168407068aa8864ea9d6 2018-10-15 & Changed int in main and spaced code & 1 & 19 & 12 \\
\hline 16 &  & 50b05095e10db4eee4bbd92773c721d7803f9f2d 2018-10-25 & Added writing and things & 13 & 505 & 1792 \\
\hline 17 &  & e0c136046374c324bbd47f0723b1055ea4290abe 2018-10-25 & modified code & 6 & 477 & 0 \\
\hline 18 &  & a87023fe93a19dbfe84e84a7374ded43bb4662db 2018-10-25 & Finished concurrency project & 2 & 156 & 73 \\
\hline 19 &  & 9ff7580e7ac32d8522936600a6cd6650cda6d7b5 2018-10-29 & made the writeup latex & 6 & 6829 & 39 \\
\hline 20 &  & 9036481e1bcef26af90e9b91e33e6f562c8177c1 2018-10-29 & Mostly done? & 10 & 670 & 57 \\
\hline 21 &  & 4729168de8c9cce8acb77f5b9baf6c5713864d04 2018-10-29 & After assignment 2 & 14 & 2830 & 438 \\
\hline 22 &  & 5f4d44aba72909a2a4a9d072166de25b30111c24 2018-11-12 & Moved code and patch files into repository & 4 & 506 & 0 \\
\hline 23 &  & 7ec298a2f7318ab4eb0c183c2af6aa13497319d7 2018-11-12 & made latex file & 4 & 511 & 0 \\
\hline 24 &  & 7295b1e68a2f97aca4fdde5a2d4fbab20bbbd444 2018-11-12 & Added how to tar & 14 & 2850 & 13 \\
\hline 25 &  & 1971a4e862078681cf793638f2f03a4d4399ea94 2018-11-26 & Assignment4 & 7 & 675 & 0 \\
\hline 26 &  & 4dff62aebc397569fb1cc08e83158cb68968cd0e 2018-11-26 & Hi there & 13 & 3466 & 44 \\
\hline 27 &  & 303c70534243944db9832f133b85a5d4dba94b02 2018-11-26 & concurrency3 & 18 & 1518 & 0 \\
\hline 28 &  & 4a2da519bd67fc5da8f16aa76419e0b634967e87 2018-11-26 & deleted unecessary tar & 1 & 0 & 0 \\
\hline 29 &  & b81d73e61646ec9c901768385dcd988c469fc6c7 2018-12-04 & Fixed it & 3 & 42 & 35 \\
\hline 30 &  &   & 0 & 0 & 0 \\
\end{longtable}


\section{Work log}


The majority of the project was completed on the due date (Monday, October 15, 2018). We saved the writeup and the coding for the producer/consumer problem for today. That made our work log real spicy for the final day. Nickoli did more of the work for the write-up and Ben did most of the coding for the prodcer/consumer problem. This worked out really well because Ben hates writing. Our group will continue this work pattern in the future. 


\end{document}
