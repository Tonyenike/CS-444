\documentclass[10pt,a4paper,article]{article}
\usepackage{times}
\usepackage{cite}
\usepackage[utf8]{inputenc}
\usepackage[T1]{fontenc}
\usepackage[english]{babel}
\usepackage{longtable, hyperref}
\usepackage{listings}
\usepackage[margin = .75 in]{geometry}
\usepackage{color}
\newcommand{\longtableendfoot}{Please continue at the next page}





{
\let\XXegroup\relax
\expandafter\def\csname XX[31\endcsname{%
  \bgroup\let\XXegroup\egroup\leavevmode\color{red}}
\expandafter\def\csname XX[1;31\endcsname{%
  \bgroup\let\XXegroup\egroup\leavevmode\bfseries\color{red}}
\expandafter\def\csname XX[32\endcsname{%
  \bgroup\let\XXegroup\egroup\leavevmode\color{green}}
\expandafter\def\csname XX[1;32\endcsname{%
  \bgroup\let\XXegroup\egroup\leavevmode\bfseries\color{green}}
\expandafter\def\csname XX[33\endcsname{%
  \bgroup\let\XXegroup\egroup\leavevmode\color{yellow}}
\expandafter\def\csname XX[1;33\endcsname{%
  \bgroup\let\XXegroup\egroup\leavevmode\bfseries\color{yellow}}
\expandafter\def\csname XX[34\endcsname{%
  \bgroup\let\XXegroup\egroup\leavevmode\color{cyan}}
\expandafter\def\csname XX[1;34\endcsname{%
  \bgroup\let\XXegroup\egroup\leavevmode\bfseries\color{cyan}}
\expandafter\def\csname XX[35\endcsname{%
  \bgroup\let\XXegroup\egroup\leavevmode\color{magenta}}
\expandafter\def\csname XX[1;35\endcsname{%
  \bgroup\let\XXegroup\egroup\leavevmode\bfseries\color{magenta}}
\expandafter\def\csname XX[36\endcsname{%
  \bgroup\let\XXegroup\egroup\leavevmode\color{blue}}
\expandafter\def\csname XX[1;36\endcsname{%
  \bgroup\let\XXegroup\egroup\leavevmode\bfseries\color{blue}}


\expandafter\def\csname XX[\endcsname{\XXegroup}
\catcode`\^^[=13
\def^^[#1m{%
\expandafter\ifx\csname XX#1\endcsname\relax
\typeout{XX#1}%
\else
\csname XX#1\endcsname
\fi}

%this is a comment
\title{Assignment 2 - I/O Elevators \& Concurrency 2}
\author{Nickoli Londura \& Benjamin Martin - Group 7 \\ Fall 2018 \\ CS 444 \\ Oregon State University}



% [language=C]

\begin{document}
\maketitle
% \tableofcontents

\begin{abstract}

\noindent This is the assignment 2 write-up for group 7 of CS 444. This assignment combines the "I/O Elevators" section and "Concurrency 2" sections on Canvas. We explore ways to make an algorithm for the LOOK I/O scheduler in the yocto kernel, and run a VM on it on the OS2 server to see if the algorithm is working properly. We also try to solve the dining philosophers problem by running concurrent threads and applying mutual exclusion (mutexes) on the threads that need to share the forks.

\end{abstract}

\newpage

\section{Write Up}
\subsection{I/O Elevators}

\textbf{1. What do you think the main point of this assignment is?}
The main point is to understand how Linux kernel schedulers are implemented. By looking into LOOK and C-LOOK I/O schedulers, we were able to learn what kind of algorithms they use in order to execute processes in a distinct order for reading and writing to the hard drive. \\


\noindent \textbf{2. How did your team approach the problem? Design decisions, algorithm, etc.}
We implemented the C-LOOK scheduler, but DONT BE DECEIVED! Everything says look, but we really mean clook. This was for ease of viewing on our own part. The C-look was easier to implement because we could just have the "elevator" run up and then reset back at the bottom once it has dispatched the uppermost request.   \\


\noindent \textbf{3. How did you ensure your solution was correct? Testing details, for instance.}
We started out by using print statements to see if the kernel was built and running properly, as well as knowing how the scheduler is adding and dispatching requests initially and at the end. We have a testing script, with a file named "yo", which was a test file with lorem ipsum text copied from here: \verb| https://loripsum.net/ | The test script was as follows: \begin{verbatim}
#!/bin/bash

i="1"

while [ $i -lt 4 ]
do
cat yo > doy
done
\end{verbatim}
It would run indefinitely until CTRL + C was hit.
\\


\noindent \textbf{4. What did you learn?}
We learned what the algorithms are for LOOK and C-LOOK I/O schedulers that are used to implement them, and also learning a bit on how they are coded. In general, we were able to learn some more about the Linux kernel, as well as getting used to working with the kernel more. We also learned about elevator algorithms and how they avoid the starvation problem, but are still more efficient than FIFO implementations. \\

\subsection{Concurrency 2}

\textbf{1. What do you think the main point of this assignment is?}
The main point was to solve a concurrency problem using a different algorithm from the last one but still using the idea of concurrency. Learning more about different concurrency algorithms will help us be more familiar with the OS. This project helped us see concurrency in a different way, while still being able to use the C language to implement it.  \\

\noindent \textbf{2. How did your team approach the problem? Design decisions, algorithm, etc.}
This problem was initially tough because we had to ensure that philosophers would grab two forks simulatenously so that they all couldn't just grab one fork and lock each other out. This issue was resolved by having a "master" lock which would disable other philosophers from taking other forks (but they can still drop!) while a philosopher is grabbing forks.\\


\noindent \textbf{3. How did you ensure your solution was correct? Testing details, for instance.}
Debugging was lots and lots of trace statements. Towards the end of the project, we had made a history and a table that would print out, showing the status of each thread and making sure the wait times are correct. We also checked mutex locks by commenting out unlock lines to see that the code would halt. This ensures that the resources are being locked by the threads.  \\

\noindent \textbf{4. What did you learn?}
We got more practice using mutexes and creating threads in C. We learned more on the idea of threads having to share resources but in a different scenario. Ideas like having all the threads share multiple items, rather than one item. We also learned about having every thread be able to run in a systematic way, where each thread is able to run when it is time to.  \\

\section{Version Control Log}

%% This file was generated by the script latex-git-log
\begin{tabular}{lp{12cm}}
  \label{tabular:legend:git-log}
  \textbf{acronym} & \textbf{meaning} \\
  V & \texttt{version} \\
  tag & \texttt{git tag} \\
  MF & Number of \texttt{modified files}. \\
  AL & Number of \texttt{added lines}. \\
  DL & Number of \texttt{deleted lines}. \\
\end{tabular}

\bigskip

% \iflanguage{ngerman}{\shorthandoff{"}}{}
\begin{longtable}{|rlllrrr|}
\hline \multicolumn{1}{|c}{\textbf{V}} & \multicolumn{1}{c}{\textbf{tag}}
& \multicolumn{1}{c}{\textbf{date}}
& \multicolumn{1}{c}{\textbf{commit message}} & \multicolumn{1}{c}{\textbf{MF}}
& \multicolumn{1}{c}{\textbf{AL}} & \multicolumn{1}{c|}{\textbf{DL}} \\ \hline
\endhead

\hline \multicolumn{7}{|r|}{\longtableendfoot} \\ \hline
\endfoot

\hline% \hline
\endlastfoot

\hline 1 &  & 57351b4b4272e2e202f9c37236e6aa529d57d48d 2018-10-04 & Initial test & 4 & 118 & 0 \\
\hline 2 &  & 457cb43d9a5ffd145692bf8048d9f67837ef5b67 2018-10-04 & Create README.md & 1 & 1 & 0 \\
\hline 3 &  & a0d31694c3643232027d1603ce90157f0aa4f7d0 2018-10-04 & Create README.md & 1 & 1 & 0 \\
\hline 4 &  & 7ed32fd0a03d4cd228a052dc6e82d18036585619 2018-10-04 & Small change & 0 & 0 & 0 \\
\hline 5 &  & 789c79dd6550916482399238babfdb32c9a9bc26 2018-10-04 & Added assignment folders & 6 & 111 & 111 \\
\hline 6 &  & 73e30d4da7ba8a16076476ad9dbc6762dabe5842 2018-10-04 & Added assignment folders & 6 & 111 & 111 \\
\hline 7 &  & f61afcd4df3637188959edd4269d9964eb01196a 2018-10-04 & Organized & 10 & 4402 & 111 \\
\hline 8 &  &   & 0 & 0 & 0 \\
\hline 9 &  &   & 0 & 0 & 0 \\
\hline 10 &  &   & 0 & 0 & 0 \\
\hline 11 &  & fa83ec19fc9d6e1c06b20b5edc4ee75f4a0d7b3a 2018-10-04 & There we go & 0 & 0 & 0 \\
\hline 12 &  & 06539913bd0fab683fe6cde3bd06cde97e9431f3 2018-10-10 & Added concurrency code & 1 & 35 & 0 \\
\hline 13 &  & b5310109d285d732ccc2527efe50f124dc0415bc 2018-10-15 & Added stuff & 6 & 57 & 4306 \\
\hline 14 &  & 840df6290e155004b32584878ee1d12810aeff5e 2018-10-15 & Finished concurrency assignment & 11 & 2102 & 57 \\
\hline 15 &  & 35987c988887860b9d57168407068aa8864ea9d6 2018-10-15 & Changed int in main and spaced code & 1 & 19 & 12 \\
\hline 16 &  & 50b05095e10db4eee4bbd92773c721d7803f9f2d 2018-10-25 & Added writing and things & 13 & 505 & 1792 \\
\hline 17 &  & e0c136046374c324bbd47f0723b1055ea4290abe 2018-10-25 & modified code & 6 & 477 & 0 \\
\hline 18 &  & a87023fe93a19dbfe84e84a7374ded43bb4662db 2018-10-25 & Finished concurrency project & 2 & 156 & 73 \\
\hline 19 &  & 9ff7580e7ac32d8522936600a6cd6650cda6d7b5 2018-10-29 & made the writeup latex & 6 & 6829 & 39 \\
\hline 20 &  & 9036481e1bcef26af90e9b91e33e6f562c8177c1 2018-10-29 & Mostly done? & 10 & 670 & 57 \\
\hline 21 &  & 4729168de8c9cce8acb77f5b9baf6c5713864d04 2018-10-29 & After assignment 2 & 14 & 2830 & 438 \\
\hline 22 &  & 5f4d44aba72909a2a4a9d072166de25b30111c24 2018-11-12 & Moved code and patch files into repository & 4 & 506 & 0 \\
\hline 23 &  & 7ec298a2f7318ab4eb0c183c2af6aa13497319d7 2018-11-12 & made latex file & 4 & 511 & 0 \\
\hline 24 &  & 7295b1e68a2f97aca4fdde5a2d4fbab20bbbd444 2018-11-12 & Added how to tar & 14 & 2850 & 13 \\
\hline 25 &  & 1971a4e862078681cf793638f2f03a4d4399ea94 2018-11-26 & Assignment4 & 7 & 675 & 0 \\
\hline 26 &  & 4dff62aebc397569fb1cc08e83158cb68968cd0e 2018-11-26 & Hi there & 13 & 3466 & 44 \\
\hline 27 &  & 303c70534243944db9832f133b85a5d4dba94b02 2018-11-26 & concurrency3 & 18 & 1518 & 0 \\
\hline 28 &  & 4a2da519bd67fc5da8f16aa76419e0b634967e87 2018-11-26 & deleted unecessary tar & 1 & 0 & 0 \\
\hline 29 &  & b81d73e61646ec9c901768385dcd988c469fc6c7 2018-12-04 & Fixed it & 3 & 42 & 35 \\
\hline 30 &  &   & 0 & 0 & 0 \\
\end{longtable}



\section{Work log}


Half of the project was completed on Thursday, October 25, 2018 for Dining Philosophers, and the other half on the due date (Monday, October 29, 2018). We saved the writeup and the coding for the I/O Elevators problem for today. That made our work log real spicy for the final day. Nickoli did most of the write-up and Ben did most of the coding for the I/O Elevators and Dining Philosophers problem. This worked out really well because Ben hates writing. Our group will continue this work pattern in the future. 


\end{document}
